\section{Introduction}
Recent years witnessed the rapid spreading of cellular networks and pervasive mobile devices (MDs). The telecommunication (Telco) data, as trace of MDs in cellular networks, has many important applications for Telco operators, e.g., city-scale Telco localization \cite{DBLP:conf/cikm/ZhuLYZZGDRZ16}, churn prediction of subscribers \cite{DBLP:conf/sigmod/HuangZYDLND0Z15} and user experience assessment \cite{DBLP:journals/tist/LuoZYDY16}. In particular, as an important complementary technique of GPS, Telco localization is aimed at recovering mobility patterns of MDs at fine grained level (e.g., 20 meters). Unlike the call detail records (CDRs), Telco localization techniques mainly focus on measurement records (MRs), which measures radio signal strengths (RSSIs) between MD and its connected cells in Telco networks. In the past years, a plenty of Telco localization methods have been proposed to improve the performance under challenging city environment \cite{DBLP:conf/icc/IbrahimY11, DBLP:conf/icc/HaraAYDZ11,DBLP:journals/tvt/IbrahimY12,DBLP:conf/infocom/RayDM16,DBLP:conf/infocom/MargoliesBBDJUV17}.

However, these localization models are suffering from missing signal strength (RSSI) values with varying degrees. Zhu \cite{DBLP:conf/cikm/ZhuLYZZGDRZ16} found that nearly 50\% of real world MR records have RSSI with only 1 or 2 cells. Ray \cite{DBLP:conf/infocom/RayDM16} proposed a localization model based on that RSSI values from neighboring cells are all missing. In the worst case, there is no RSSI at all, left only the serving cells in MR records \cite{DBLP:conf/gis/PereraBKB15}. The missing data problem significantly deteriorates the performance of Telco localization. There are two main reasons of missing values in MR records. One is that the mobile phones do not provide API to access neighboring cells. The other is that the RSSI is lost, due to communication failure or data corruption. Consequently, it is desired to develop a methodology with high completion performance to estimate the missing data.

The most frequently used methods for data completion are interpolation, statistics and nearest neighbors \cite{DBLP:journals/bioinformatics/TroyanskayaCSBHTBA01}. These methods simply fill missing values by part of the data set, do not generate high quality complete data. . The recent proposed algorithms built on deep learning (e.g. denoising autoencoders (DAE)) \cite{DBLP:conf/pakdd/GondaraW18} learn the conditional probability from complete parts to missing parts, based on the whole data set. Given existing data completion methods, there are two major challenges in applying them directly for MR data.

(1) The existing methods cannot capture the correlation of cell locations and associated RSSI, which is most important in completing missing RSSI.

(2) Due to complex RF propagation phenomena (e.g., multipath and non-line-of-sight), the signal strengths often fluctuate in a wide range. Noise caused by such fluctuation can hurt the inference of missing RSSI.

To this end, we propose TelcoGAN that generalizes Generative Adversarial Nets \cite{DBLP:conf/nips/GoodfellowPMXWOCB14} with  to generate complete data set.

Besides, there is a major difference in our problem. The missing MR data to complete is collected without location labels (e.g., GPS coordinates) due to GPS issues \cite{DBLP:conf/mdm/ZhangRYZY17}. However, by retrieving locations from location-based services, is is available in historical collected MR data \cite{DBLP:journals/imwut/HuangLWCXZ17}. How to exploit these labels to help the generation of missing RSSI remains challenging. 