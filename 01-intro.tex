\section{Introduction}
Recent years witnessed the rapid spreading of cellular networks and pervasive mobile devices (MDs). The telecommunication (Telco) data, as trace of MDs in cellular networks, has many important applications for Telco operators, e.g., city-scale Telco localization \cite{DBLP:conf/cikm/ZhuLYZZGDRZ16}, churn prediction of subscribers \cite{DBLP:conf/sigmod/HuangZYDLND0Z15} and user experience assessment \cite{DBLP:journals/tist/LuoZYDY16}. In particular, as an important complementary technique of GPS, Telco localization is aimed at recovering mobility patterns of MDs at fine grained level (e.g., 20 meters). Unlike the call detail records (CDRs), Telco localization techniques mainly focus on measurement records (MRs), which measures radio signal strengths (RSSIs) between MD and its connected cells in Telco networks. In the past years, a plenty of Telco localization methods have been proposed to improve the performance under challenging city environment \cite{DBLP:conf/icc/IbrahimY11, DBLP:conf/icc/HaraAYDZ11,DBLP:journals/tvt/IbrahimY12,DBLP:conf/infocom/RayDM16,DBLP:conf/infocom/MargoliesBBDJUV17}.

However, these localization models are suffering from missing signal strength (RSSI) values. Zhu \cite{DBLP:conf/cikm/ZhuLYZZGDRZ16} found that nearly 50\% of real world MR records have RSSI with only 1 or 2 cells. Ray \cite{DBLP:conf/infocom/RayDM16} proposed a localization model based on that RSSI values from neighboring cells are all missing. The missing data problem significantly deteriorates the performance of Telco localization. There are two main reasons of missing values in MR records. One is that the mobile phones do not provide API to access neighboring cells. The other is that the RSSI is lost, due to communication failure or data corruption. Consequently, it is desired to develop a methodology with high completion performance to estimate the missing data.

The Telco missing data completion work faces two main challenges. 1) \emph{Complex internal relationship}: Due to complicated Telco operation design, the internal relationship of MR data is sophisticated and nonlinear. 2) \emph{Noisy signal strength}: The challenging RF propagation phenomena (e.g., multipath and non-line-of-sight) causes noise signal strength values. Noise caused by such fluctuation can hurt the inference of missing RSSI.

Given the aforementioned challenges, existing methods for missing data completion do not make high quality results. The most frequently used methods for data completion are interpolation, statistics and nearest neighbors \cite{DBLP:journals/bioinformatics/TroyanskayaCSBHTBA01}. These methods simply fill missing values by part of the data set, do not generate high quality complete data. . The recent proposed algorithms built on deep learning (e.g. denoising autoencoders (DAE)) \cite{DBLP:conf/pakdd/GondaraW18} learn the conditional probability from complete parts to missing parts, based on the whole data set.

%Besides, there is a major difference in our problem. The missing MR data to complete is collected without location labels (e.g., GPS coordinates) due to GPS issues \cite{DBLP:conf/mdm/ZhangRYZY17}. However, by retrieving locations from location-based services, it is available in historical collected MR data \cite{DBLP:journals/imwut/HuangLWCXZ17}. How to exploit these labels to help the generation of missing RSSI remains challenging.

Meanwhile with recent breakthroughs in deep learning, Generative Adversarial Nets (GANs), as a powerful technique for generative modelling, has shown impressive results in realistic data generation \cite{DBLP:conf/nips/GoodfellowPMXWOCB14,DBLP:conf/cvpr/LedigTHCCAATTWS17}. The GANs plays a game between two networks: a generator that produces completed MR data given MR missing data and a discriminator that produces probability distribution between the completed and real data \cite{DBLP:conf/nips/GoodfellowPMXWOCB14}. Compared with traditional machine learning models, the competing process between two networks are better at learning sophisticated internal correlation of data.

To this end, we propose TelcoGAN that built upon GANs to address the MR missing data problem. TelcoGAN takes advantage of available GPS location labels in training data set (e.g., retrieving location corrdinates from location-based services \cite{DBLP:journals/imwut/HuangLWCXZ17}), to stabilize the training process and lead to better completion performance. Our research makes two main contributions:

\begin{itemize}
  \item We propose TelcoGAN which cooperates a \emph{serving-centric space} and a \emph{localizer network} to produce high quality complete MR data. The serving-centric space component helps to learn the internal correlation of MR. The localizer network utilizes available GPS location labels to guide the adversarial process towards better results. In addition, we adopt a hybrid loss trick which combines mean squared loss and adversarial loss to further improve the performance.
  \item Evaluations on two real-world MR data sets show that our model achieves better performance than state-of-arts and the model trained on a spatial domain can improve the performance of another one.
\end{itemize}

%The goal of generator is to complete MR missing data, and the goal of discriminator is to distinguish between real MR data and completed MR data. The discriminator is trained to maximize classification accuracy (real data/completed data), and the generator is trained to minimize the discriminator's classification accuracy. Thus, the two 