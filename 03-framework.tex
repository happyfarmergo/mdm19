\section{TelcoGAN Model}
In this section, we introduce our TelcoGAN model for Telco missing data completion. The descriptions are separated into three parts. First, we describe the design of TelcoGAN model as well as the interaction of different components. Second, we give the details of relative coordinate space and explore the real-world relationship between connected cells and RSSI. Next, the designs of generator, discriminator and localizer is introduced respectively. Finally, we give TelcoGAN's training process.

\begin{figure}
  \centering
  \includegraphics[width=9cm]{pics/framework.pdf}
  \caption{The framework of TelcoGAN}\label{fig:framework}
\end{figure}


\subsection{Overview of TelcoGAN}
The TelcoGAN model consists of pre-processing step and three basic components. The framework of TelcoGAN is illustrated as Fig. \ref{fig:framework}.

In the pre-processing step, due to sparse extensive cell locations in MR data, we propose to apply a serving-centric space. Then the sparse global coordinate-based distribution of MR data is transformed into a dense relative coordinate-based distribution. Thus TelcoGAN model can better capture the internal correlation of MR data.

The adversarial learning step consists of three interacting components described as follows:

(1) \textbf{Generator} $\bar{\textbf{m}}\sim$ G(\textbf{z},$\hat{\textbf{m}}$, \textbf{b}): The generator takes a random vector, an incomplete MR matrix and corresponding binary vector as input, and generates a completed MR matrix $\bar{\textbf{m}}$ that fools the discriminator as well as makes localizer produce accurate predictions.

(2) \textbf{Discriminator} D($\bar{\textbf{m}}/\textbf{m}$): The discriminator takes either a complete MR sample or a real MR sample as input, and gives each sample the probability over two categories (real/completed).

(3) \textbf{Localizer} L($\bar{\textbf{m}}/\textbf{m}$, $\textbf{l}$): The localizer takes a pair of MR sample and corresponding location label as input. It tries to predict the position of MR sample and minimize the localization loss.

The intuition of how our TelcoGAN model can generate high quality MR data for Telco localization is as follows. The generator tries to recover complete MR data samples based on observed variables to fool the discriminator; The discriminator distinguishes input data samples and computes probability distribution that the samples comes from real data or generated data; The localizer predicts locations of MR samples and produces a score for each sample that reflects its quality. During the adversarial game between the generator and the discriminator, the localizer can guide the optimization towards better data quality by utilizing available location labels. When the training reaches the optimality, the generator will have learnt the mapping from incomplete MR data to complete MR data.

We choose to implement these three components as neural networks. We will discuss their detailed structures for generating complete MR data in the following subsections.

\subsection{Serving-centric Space}
Here we construct a serving-centric space for MR data to better learn the internal relationship.

As seen in Table. \ref{tab:mr}, each MR record has a serving cell. According to Telco operations \cite{DBLP:conf/infocom/RayDM16}, the serving cell is selected from these nearby cells with good connection, which means close distance to mobile device. Thus a serving cell indicates a specific spatial domain. Given the total MR records with extensive global spatial domain, we group them as multiple local spatial domains by the serving cell. Then the whole city-scale area is divided into multiple small spatial domains.

<<<<<<< HEAD
As seen in Table. \ref{tab:mr}, among 6 connected cells, . According to Telco operations \cite{DBLP:conf/infocom/RayDM16}, the serving cell is selected from these nearby cells with good connection, which indicates close distance to mobile device. Given the total MR records with extensive global spatial domain, we group them as multiple local spatial domains by same serving cell. Then the whole city-scale area is divided into multiple small spatial domains.
=======
Based on the division above, we propose a \emph{serving-centric space}, to transform the sparse global spatial distribution into a dense local one. In Fig. , a spatial domain is an area centered on the serving cell. Provided the cell tower database from Telco operators, we can obtain the GPS coordinate location of each cell. Based on the cell location data, we can do coordinate conversion as follows. Suppose the GPS coordinate (longitude, latitude) of the serving cell is $(x_0,y_0)$, those of the neighboring cells are $\{(x_i,y_i)\}$($1\leq i\leq 5$). Under the serving-centric space, the new coordinates are $(0,0)$ and $\{(x_i-x_0,y_i-y_0)\}$($1\leq i\leq 5$) for the serving cell and the neighboring cells respectively.
>>>>>>> b02095b0baba0ec99676fa4c5448097aa6adc555

The \textbf{rational} behind serving-centric space is not difficult to understand: suppose normal cells and similar RF environment, in principle, the received signal strengths are only decided by the relative locations of connected cells and the MD. The serving-centric space offers advantages as: 1) reveal the true internal relationship of MR data and 2) knowledge learned from a domain can transfer to another domain and improve data quality.

<<<<<<< HEAD
The \textbf{rational} behind \emph{Serving-centric space} is not difficult to understand: suppose normal cells and similar RF environment, in principle, the received signal strengths are only decided by the relative locations of all cells and the connected MD. The \emph{Serving-centric space} offers advantages as: 1) reveal true internal relationship of MR data and 2) knowledge learned from a spatial domain can transfer to another domain and improve data quality.
=======
>>>>>>> b02095b0baba0ec99676fa4c5448097aa6adc555

\subsection{TelcoGAN Generator}
The goal here is to generate completed MR data by modelling the conditional probability of completed MR data given missing MR data.

The generator G takes a MR record $\hat{m}$ with missing RSSI, a random noise vector $z$ and an indicator vector $b$ as input, and outputs a complete MR record $\bar{m}$. The generator G is realized as convolutional neural networks (CNN), due to its capability of learning complex interactions. The process of generator G can be formulated by:

\begin{equation}\label{eq:gen}
  \tilde{m} = G(\hat{m}, (1-b)\odot z)
\end{equation}

where $\odot$ denotes element-wise multiplication. $(1-b)\odot z$ indicates random values to fill in missing RSSI in $\hat{m}$, that is, G uses random noise vector to replace the missingness for initialization of incomplete MR. Then G learns the conditional probability $P(\tilde{m}|\hat{m}, b, z)$ by convolutional nets.

\subsection{TelcoGAN Discriminator}
The TelcoGAN discriminator aims to differentiate completed MR records from the real ones by modelling the distribution of MR data.

The discriminator D takes either a completed MR record or a real MR record as input, and outputs the probability over two categories (real/completed). The D learns the probability of MR data $P(m)$ by full-convolution structure.

\subsection{TelcoGAN Localizer}


\subsection{TelcoGAN Training}
