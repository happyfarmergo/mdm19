\section{Preliminaries}
In this section, we first give a detailed description of MR data and some basic notations, then followed by the problem definition and overview of GANs \cite{DBLP:conf/nips/GoodfellowPMXWOCB14} with its variants.

\subsection{Data Description and Notations}
Telco localization techniques mostly focus on MR data, which are generated when MDs connect to nearby cell towers in Telco networks. Generally, the MR data can be categorized into two aspects: (1) connected cells data including cell ids, GPS locations; (2) continuous signal strength data, such as RSRP, RSSI. Table \ref{tab:mr} shows an example of MR record in 4G LTE network. A piece of MR record contains up to 6 nearby cells (eNodeBID and CellID) and radio signal strength indicators (RSSI) for each. Besides, it also marks a user ID (IMSI: International Mobile Subscriber identification Number) and connection time stamp (MRTime).

For the rest of this paper, a MR record and associated GPS label are denoted by $m$ and $l$ respectively. For a MR record vector $m$, it consists of cell id vector $c$, cell coordinate vector $d$ and RSSI vector $r$ with equal length as 6. Note that to protect privacy of involved users, we delete IMSI to reduce risk.

\begin{table}\scriptsize
\caption{An example of 4G LTE MR record}\label{tab:mr}
  \centering
  \begin{tabular}{|l|l||l|l|}
  \hline
  \textbf{Field}    & \textbf{Value}                 & \textbf{Field}    & \textbf{Value}   \\ \hline \hline
  MRTime            & {\textbf{2017/5/31 14:12:06}}  & IMSI              & \textbf{***012}  \\ \hline
  Serving\_eNodeBID & \textbf{99129}                 & Serving\_CellID   & \textbf{1}       \\ \hline
  eNodeBID\_1       & \textbf{99129}                 & CellID\_1         & \textbf{1}       \\ \hline
  RSRP\_1           & \textbf{-93.26}                & RSSI\_1           & \textbf{-67.18}  \\ \hline
  ...               & ...                            & ...               & ...              \\ \hline
  eNodeBID\_6       & \textbf{99145}                 & CellID\_6         & \textbf{5}       \\ \hline
  RSRP\_6           & \textbf{-90.02}                & RSSI\_6           & \textbf{-50.92}  \\ \hline
  \end{tabular}
\end{table}

\subsection{Problem Definition}
The MR missing data problem refers to missing of RSSI from neighboring cells. We assume that RSSI from any neighboring cell can be missing randomly (i.e., from $r_2$ to $r_6$). In addition, the binary vector $b=\{b_i\}_{i=1}^6$ taken values in $\{0, 1\}$ indicates which part of RSSI vector $r$ could be lost. Thus, an incomplete RSSI vector $\hat{r}$ can be defined as follow:

\begin{equation}\label{eq:rssi}
\hat{r_i}=\left\{
\begin{aligned}
&r_i, & \text{if}\enspace b_i=1 \\
&nan,  & \text{otherwise}
\end{aligned}
\right.
\end{equation}

Note that we can retrieve binary vector $b$ from $\hat{r}$.

The detail of the problem studied in this paper is described in Problem \ref{prob:miss}.

\begin{problem}\label{prob:miss}
  \textbf{Telco Missing Data Completion}: For an incomplete MR record $m$ with RSSI vector $\hat{r}$, we aim at filling the missingness of $\hat{r}$ and generating $r$, given the cell id vector $c$ and cell coordinate vector $d$.
\end{problem}

\subsection{Basic of GANs}
The key of Generative Adversarial Nets (GANs) \cite{DBLP:conf/nips/GoodfellowPMXWOCB14} is to play a competing game between two networks. The generator network $G$ takes noise vector $z$ as input and generates fake data. The discriminator network $D$ takes a real data sample or fake data sample as input and try to classify accurately. In contrast, the generator $G$ try to generate realistic data to fool discriminator $D$. Hence, the two networks $G$ and $D$ play a minimax game which can be formulated as:

\begin{equation}\label{eq:gan}
  \min\limits_G \max\limits_D \mathbb{E}_{\textbf{x}\sim p_(\textbf{x})}[\log D(\textbf{x})]+\mathbb{E}_{\textbf{z}\sim p( \textbf{z})}[\log(1-D(G(\textbf{z})))]
\end{equation}

where $p_(\textbf{x})$ denotes real data distribution, $p_(\textbf{z})$ is noise distribution such as the uniform distribution or the normal distribution. 